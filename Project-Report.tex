% Options for packages loaded elsewhere
\PassOptionsToPackage{unicode}{hyperref}
\PassOptionsToPackage{hyphens}{url}
%
\documentclass[
]{article}
\usepackage{amsmath,amssymb}
\usepackage{lmodern}
\usepackage{iftex}
\ifPDFTeX
  \usepackage[T1]{fontenc}
  \usepackage[utf8]{inputenc}
  \usepackage{textcomp} % provide euro and other symbols
\else % if luatex or xetex
  \usepackage{unicode-math}
  \defaultfontfeatures{Scale=MatchLowercase}
  \defaultfontfeatures[\rmfamily]{Ligatures=TeX,Scale=1}
\fi
% Use upquote if available, for straight quotes in verbatim environments
\IfFileExists{upquote.sty}{\usepackage{upquote}}{}
\IfFileExists{microtype.sty}{% use microtype if available
  \usepackage[]{microtype}
  \UseMicrotypeSet[protrusion]{basicmath} % disable protrusion for tt fonts
}{}
\makeatletter
\@ifundefined{KOMAClassName}{% if non-KOMA class
  \IfFileExists{parskip.sty}{%
    \usepackage{parskip}
  }{% else
    \setlength{\parindent}{0pt}
    \setlength{\parskip}{6pt plus 2pt minus 1pt}}
}{% if KOMA class
  \KOMAoptions{parskip=half}}
\makeatother
\usepackage{xcolor}
\IfFileExists{xurl.sty}{\usepackage{xurl}}{} % add URL line breaks if available
\IfFileExists{bookmark.sty}{\usepackage{bookmark}}{\usepackage{hyperref}}
\hypersetup{
  pdftitle={Project-QM},
  pdfauthor={Karan, Rohit, Shubham},
  hidelinks,
  pdfcreator={LaTeX via pandoc}}
\urlstyle{same} % disable monospaced font for URLs
\usepackage[margin=1in]{geometry}
\usepackage{color}
\usepackage{fancyvrb}
\newcommand{\VerbBar}{|}
\newcommand{\VERB}{\Verb[commandchars=\\\{\}]}
\DefineVerbatimEnvironment{Highlighting}{Verbatim}{commandchars=\\\{\}}
% Add ',fontsize=\small' for more characters per line
\usepackage{framed}
\definecolor{shadecolor}{RGB}{248,248,248}
\newenvironment{Shaded}{\begin{snugshade}}{\end{snugshade}}
\newcommand{\AlertTok}[1]{\textcolor[rgb]{0.94,0.16,0.16}{#1}}
\newcommand{\AnnotationTok}[1]{\textcolor[rgb]{0.56,0.35,0.01}{\textbf{\textit{#1}}}}
\newcommand{\AttributeTok}[1]{\textcolor[rgb]{0.77,0.63,0.00}{#1}}
\newcommand{\BaseNTok}[1]{\textcolor[rgb]{0.00,0.00,0.81}{#1}}
\newcommand{\BuiltInTok}[1]{#1}
\newcommand{\CharTok}[1]{\textcolor[rgb]{0.31,0.60,0.02}{#1}}
\newcommand{\CommentTok}[1]{\textcolor[rgb]{0.56,0.35,0.01}{\textit{#1}}}
\newcommand{\CommentVarTok}[1]{\textcolor[rgb]{0.56,0.35,0.01}{\textbf{\textit{#1}}}}
\newcommand{\ConstantTok}[1]{\textcolor[rgb]{0.00,0.00,0.00}{#1}}
\newcommand{\ControlFlowTok}[1]{\textcolor[rgb]{0.13,0.29,0.53}{\textbf{#1}}}
\newcommand{\DataTypeTok}[1]{\textcolor[rgb]{0.13,0.29,0.53}{#1}}
\newcommand{\DecValTok}[1]{\textcolor[rgb]{0.00,0.00,0.81}{#1}}
\newcommand{\DocumentationTok}[1]{\textcolor[rgb]{0.56,0.35,0.01}{\textbf{\textit{#1}}}}
\newcommand{\ErrorTok}[1]{\textcolor[rgb]{0.64,0.00,0.00}{\textbf{#1}}}
\newcommand{\ExtensionTok}[1]{#1}
\newcommand{\FloatTok}[1]{\textcolor[rgb]{0.00,0.00,0.81}{#1}}
\newcommand{\FunctionTok}[1]{\textcolor[rgb]{0.00,0.00,0.00}{#1}}
\newcommand{\ImportTok}[1]{#1}
\newcommand{\InformationTok}[1]{\textcolor[rgb]{0.56,0.35,0.01}{\textbf{\textit{#1}}}}
\newcommand{\KeywordTok}[1]{\textcolor[rgb]{0.13,0.29,0.53}{\textbf{#1}}}
\newcommand{\NormalTok}[1]{#1}
\newcommand{\OperatorTok}[1]{\textcolor[rgb]{0.81,0.36,0.00}{\textbf{#1}}}
\newcommand{\OtherTok}[1]{\textcolor[rgb]{0.56,0.35,0.01}{#1}}
\newcommand{\PreprocessorTok}[1]{\textcolor[rgb]{0.56,0.35,0.01}{\textit{#1}}}
\newcommand{\RegionMarkerTok}[1]{#1}
\newcommand{\SpecialCharTok}[1]{\textcolor[rgb]{0.00,0.00,0.00}{#1}}
\newcommand{\SpecialStringTok}[1]{\textcolor[rgb]{0.31,0.60,0.02}{#1}}
\newcommand{\StringTok}[1]{\textcolor[rgb]{0.31,0.60,0.02}{#1}}
\newcommand{\VariableTok}[1]{\textcolor[rgb]{0.00,0.00,0.00}{#1}}
\newcommand{\VerbatimStringTok}[1]{\textcolor[rgb]{0.31,0.60,0.02}{#1}}
\newcommand{\WarningTok}[1]{\textcolor[rgb]{0.56,0.35,0.01}{\textbf{\textit{#1}}}}
\usepackage{graphicx}
\makeatletter
\def\maxwidth{\ifdim\Gin@nat@width>\linewidth\linewidth\else\Gin@nat@width\fi}
\def\maxheight{\ifdim\Gin@nat@height>\textheight\textheight\else\Gin@nat@height\fi}
\makeatother
% Scale images if necessary, so that they will not overflow the page
% margins by default, and it is still possible to overwrite the defaults
% using explicit options in \includegraphics[width, height, ...]{}
\setkeys{Gin}{width=\maxwidth,height=\maxheight,keepaspectratio}
% Set default figure placement to htbp
\makeatletter
\def\fps@figure{htbp}
\makeatother
\setlength{\emergencystretch}{3em} % prevent overfull lines
\providecommand{\tightlist}{%
  \setlength{\itemsep}{0pt}\setlength{\parskip}{0pt}}
\setcounter{secnumdepth}{-\maxdimen} % remove section numbering
\ifLuaTeX
  \usepackage{selnolig}  % disable illegal ligatures
\fi

\title{Project-QM}
\author{Karan, Rohit, Shubham}
\date{14/03/2022}

\begin{document}
\maketitle

\hypertarget{introduction}{%
\subsubsection{Introduction}\label{introduction}}

In this study, we have evaluated the impact of weather on retail sales
in the major cities of Canada. We leveraged the data of 3 major retail
players- Walmart, Costco and The Wholesale Food for analysing how the
weather affects thei retail sales.\\
1.Role of Data It is difficult to anticipate daily weather in advance
but it has an immediate impact on daily business and sales.The use of
weather forecasts to reduce demand uncertainty is restricted to
retailers able to adjust their supply chain activities within two weeks.
Retail sales being the downstream component of supply chain has

1.1 Seasonality Weather such as precipitation(snow and rain) can
significantly influence shopper's decision. It is obscure how consumer
will react to weather conditions, The shoppers may choose to use good
weather as a reason for engaging in outdoor activities and thus postpone
or forgo purchases. On the other hand they might consider shopping in
clear and sunny weather. \#Brick and Mortar Holiday season can have pos

1.2. Psychological Effect People in good mood are more likely to spend
more money and self reward themselves.Weather can have negative effect
on an individual's mood. Bad weather can have a negative psychological
impact on shopper's willingness to visit retail stores. It might cause
them to shop online or postpone the plan of shopping. \#BrickandMortar.
Psychological effects may lead to a change in shopping habits, e.g.,
bulk buying to hoard supplies. For example, lower prices during an
extreme weather occurrence might trigger stock-ups (Gauri et al., 2017).
Lower prices during extreme weather occurrences are common in retail
settings (Beutel and Minner, 2012). It has been shown that the weather
at the time of purchase heavily affects customers' decisions for
purchases, e.g., advance sales of outdoor events, when the weather at
the time of purchase should be irrelevant.

======= 1.Role of Weather 1.1. Seasonality\\
1.2. Psychological Effect

\hypertarget{objective}{%
\subsubsection{Objective}\label{objective}}

The objective of this research paper is to analyze the impact of weather
on retail sales in 3 major Canadian cities- Montreal, Toronto and
Vancouver. 2. Footfall is our proxy variable. We are assessing the
retail sales through the footfall data of stores which acts as our proxy
variable. (Graham, 2017; Makkar, 2020).

\begin{enumerate}
\def\labelenumi{\arabic{enumi}.}
\setcounter{enumi}{2}
\item
  Weather Variables -\textgreater{} snow, precipitation, sunlight \#\#\#
  Literature Review Weather impact on retail sales has been studied for
  more than 50 years now (Steele, 1951), but still there are only a
  handful of studies that provide sophisticated econometric analysis of
  weather effect. Starr-McCluer (2000) examined the effects of
  temperature on total US retail sales. \#
\item
  Past Articles -1.1 and 1.2
\item
  How our research is novel?
\item
  How footfall is related to Sales?
\item
  Hypotheses : 3/4 depending on independent variables.
\item
  Control Variables and there importances--- refer to articles

  \begin{enumerate}
  \def\labelenumii{\alph{enumii}.}
  \tightlist
  \item
    CPI b. unemployability c.~disposable income d.~consumer confidence
    Talk about all sections of paper- what's done in which section.
  \end{enumerate}
\end{enumerate}

\hypertarget{data-collection-and-preprocessing}{%
\subsubsection{Data Collection and
preprocessing}\label{data-collection-and-preprocessing}}

safegraph statista.ca Python/ R preprocessing

\hypertarget{data-preperation}{%
\section{Data Preperation}\label{data-preperation}}

\#pre-holday, post holiday, day of the week, long weekend \#\#\# EDA

\hypertarget{modelling}{%
\subsubsection{Modelling}\label{modelling}}

\begin{Shaded}
\begin{Highlighting}[]
\FunctionTok{library}\NormalTok{(plm)}
\end{Highlighting}
\end{Shaded}

\begin{verbatim}
## 
## Attaching package: 'plm'
\end{verbatim}

\begin{verbatim}
## The following objects are masked from 'package:dplyr':
## 
##     between, lag, lead
\end{verbatim}

\begin{Shaded}
\begin{Highlighting}[]
\NormalTok{base\_model\_random }\OtherTok{\textless{}{-}} \FunctionTok{plm}\NormalTok{(visits\_by\_day }\SpecialCharTok{\textasciitilde{}}\NormalTok{ ConsumerConfidence }\SpecialCharTok{+}\NormalTok{ cpi }\SpecialCharTok{+}\NormalTok{ disposableIncome }\SpecialCharTok{+}\NormalTok{ unemployability, }\AttributeTok{data =}\NormalTok{ my\_data, }\AttributeTok{index =} \FunctionTok{c}\NormalTok{(}\StringTok{"postal\_code"}\NormalTok{, }\StringTok{"date"}\NormalTok{), }\AttributeTok{model =} \StringTok{"within"}\NormalTok{)}
\FunctionTok{summary}\NormalTok{(base\_model\_random)}
\end{Highlighting}
\end{Shaded}

\begin{verbatim}
## Oneway (individual) effect Within Model
## 
## Call:
## plm(formula = visits_by_day ~ ConsumerConfidence + cpi + disposableIncome + 
##     unemployability, data = my_data, model = "within", index = c("postal_code", 
##     "date"))
## 
## Unbalanced Panel: n = 17, T = 637-1022, N = 16975
## 
## Residuals:
##      Min.   1st Qu.    Median   3rd Qu.      Max. 
## -5828.826  -414.946   -81.273   364.350 20482.798 
## 
## Coefficients:
##                       Estimate  Std. Error t-value  Pr(>|t|)    
## ConsumerConfidence  -2.4034141   0.5196546 -4.6250 3.773e-06 ***
## cpi                -56.4643256   7.9743164 -7.0808 1.490e-12 ***
## disposableIncome     0.0565051   0.0021907 25.7927 < 2.2e-16 ***
## unemployability     -0.3052353   0.0772846 -3.9495 7.863e-05 ***
## ---
## Signif. codes:  0 '***' 0.001 '**' 0.01 '*' 0.05 '.' 0.1 ' ' 1
## 
## Total Sum of Squares:    2.3075e+10
## Residual Sum of Squares: 2.1591e+10
## R-Squared:      0.064286
## Adj. R-Squared: 0.063182
## F-statistic: 291.194 on 4 and 16954 DF, p-value: < 2.22e-16
\end{verbatim}

\begin{Shaded}
\begin{Highlighting}[]
\NormalTok{data }\OtherTok{\textless{}{-}}\NormalTok{ data }\SpecialCharTok{\%\textgreater{}\%} \FunctionTok{replace\_na}\NormalTok{(}\FunctionTok{list}\NormalTok{(}\AttributeTok{spd\_max\_gust=}\DecValTok{0}\NormalTok{))}
\NormalTok{data }\OtherTok{\textless{}{-}}\NormalTok{ data }\SpecialCharTok{\%\textgreater{}\%} \FunctionTok{replace\_na}\NormalTok{(}\FunctionTok{list}\NormalTok{(}\AttributeTok{snow\_grnd =} \DecValTok{0}\NormalTok{))}
\NormalTok{data }\OtherTok{\textless{}{-}}\NormalTok{ data }\SpecialCharTok{\%\textgreater{}\%} \FunctionTok{replace\_na}\NormalTok{(}\FunctionTok{list}\NormalTok{(}\AttributeTok{total\_rain =} \DecValTok{0}\NormalTok{))}
\NormalTok{data }\OtherTok{\textless{}{-}}\NormalTok{ data }\SpecialCharTok{\%\textgreater{}\%} \FunctionTok{replace\_na}\NormalTok{(}\FunctionTok{list}\NormalTok{(}\AttributeTok{total\_snow =} \DecValTok{0}\NormalTok{))}
\NormalTok{data }\OtherTok{\textless{}{-}}\NormalTok{ data }\SpecialCharTok{\%\textgreater{}\%} \FunctionTok{replace\_na}\NormalTok{(}\FunctionTok{list}\NormalTok{(}\AttributeTok{total\_precip =} \DecValTok{0}\NormalTok{))}
\NormalTok{data }\OtherTok{\textless{}{-}}\NormalTok{ data }\SpecialCharTok{\%\textgreater{}\%} \FunctionTok{replace\_na}\NormalTok{(}\FunctionTok{list}\NormalTok{(}\AttributeTok{visits\_by\_day =} \DecValTok{0}\NormalTok{))}

\NormalTok{my\_data }\OtherTok{\textless{}{-}}\NormalTok{ data }\SpecialCharTok{\%\textgreater{}\%} \FunctionTok{group\_by}\NormalTok{(postal\_code, Year, Month) }\SpecialCharTok{\%\textgreater{}\%} \FunctionTok{summarize}\NormalTok{(}\AttributeTok{visits\_by\_day =} \FunctionTok{sum}\NormalTok{(visits\_by\_day),}
                                                                     \AttributeTok{mean\_temp =} \FunctionTok{mean}\NormalTok{(mean\_temp),}
                                                                     \AttributeTok{ConsumerConfidence =} \FunctionTok{mean}\NormalTok{(ConsumerConfidence),}
                                                                     \AttributeTok{cpi =} \FunctionTok{mean}\NormalTok{(cpi),}
                                                                     \AttributeTok{disposableIncome =} \FunctionTok{mean}\NormalTok{(disposableIncome),}
                                                                     \AttributeTok{unemployability =} \FunctionTok{mean}\NormalTok{(unemployability),}
                                                                     \AttributeTok{total\_snow =} \FunctionTok{sum}\NormalTok{(total\_snow),}
                                                                     \AttributeTok{total\_rain =} \FunctionTok{sum}\NormalTok{(total\_rain),}
                                                                     \AttributeTok{total\_precip =} \FunctionTok{sum}\NormalTok{(total\_precip),}
                                                                     \AttributeTok{snow\_grnd =} \FunctionTok{mean}\NormalTok{(snow\_grnd),}
                                                                     \AttributeTok{spd\_max\_gust =} \FunctionTok{mean}\NormalTok{(spd\_max\_gust)}
\NormalTok{                                                                     )}
\end{Highlighting}
\end{Shaded}

\begin{verbatim}
## `summarise()` has grouped output by 'postal_code', 'Year'. You can override
## using the `.groups` argument.
\end{verbatim}

\begin{Shaded}
\begin{Highlighting}[]
\NormalTok{lm.full }\OtherTok{=} \FunctionTok{lm}\NormalTok{(visits\_by\_day }\SpecialCharTok{\textasciitilde{}}\NormalTok{ ., }\AttributeTok{data =}\NormalTok{ my\_data[,}\DecValTok{4}\SpecialCharTok{:}\DecValTok{13}\NormalTok{])}
\NormalTok{lm.step }\OtherTok{=} \FunctionTok{step}\NormalTok{(lm.full, }\AttributeTok{direction =} \StringTok{"backward"}\NormalTok{)}
\end{Highlighting}
\end{Shaded}

\begin{verbatim}
## Start:  AIC=12866.12
## visits_by_day ~ mean_temp + ConsumerConfidence + cpi + disposableIncome + 
##     unemployability + total_snow + total_rain + total_precip + 
##     snow_grnd
## 
##                      Df  Sum of Sq        RSS   AIC
## - disposableIncome    1 3.6830e+08 1.5719e+12 12864
## - unemployability     1 1.7207e+09 1.5732e+12 12865
## - total_precip        1 2.9115e+09 1.5744e+12 12865
## - ConsumerConfidence  1 3.3888e+09 1.5749e+12 12865
## - total_rain          1 3.8231e+09 1.5753e+12 12866
## <none>                             1.5715e+12 12866
## - total_snow          1 6.9868e+09 1.5785e+12 12867
## - snow_grnd           1 1.6028e+10 1.5875e+12 12870
## - mean_temp           1 2.0958e+10 1.5924e+12 12872
## - cpi                 1 4.9716e+10 1.6212e+12 12883
## 
## Step:  AIC=12864.26
## visits_by_day ~ mean_temp + ConsumerConfidence + cpi + unemployability + 
##     total_snow + total_rain + total_precip + snow_grnd
## 
##                      Df  Sum of Sq        RSS   AIC
## - total_precip        1 3.0849e+09 1.5749e+12 12863
## - ConsumerConfidence  1 4.0828e+09 1.5759e+12 12864
## - total_rain          1 4.1208e+09 1.5760e+12 12864
## <none>                             1.5719e+12 12864
## - total_snow          1 6.9641e+09 1.5788e+12 12865
## - snow_grnd           1 1.5817e+10 1.5877e+12 12868
## - mean_temp           1 2.1039e+10 1.5929e+12 12870
## - unemployability     1 3.9336e+10 1.6112e+12 12877
## - cpi                 1 7.6842e+10 1.6487e+12 12890
## 
## Step:  AIC=12863.42
## visits_by_day ~ mean_temp + ConsumerConfidence + cpi + unemployability + 
##     total_snow + total_rain + snow_grnd
## 
##                      Df  Sum of Sq        RSS   AIC
## - total_rain          1 1.4034e+09 1.5763e+12 12862
## - ConsumerConfidence  1 3.6786e+09 1.5786e+12 12863
## - total_snow          1 4.2897e+09 1.5792e+12 12863
## <none>                             1.5749e+12 12863
## - snow_grnd           1 1.5572e+10 1.5905e+12 12867
## - mean_temp           1 1.9232e+10 1.5942e+12 12869
## - unemployability     1 3.6425e+10 1.6114e+12 12875
## - cpi                 1 7.5283e+10 1.6502e+12 12889
## 
## Step:  AIC=12861.95
## visits_by_day ~ mean_temp + ConsumerConfidence + cpi + unemployability + 
##     total_snow + snow_grnd
## 
##                      Df  Sum of Sq        RSS   AIC
## - ConsumerConfidence  1 3.5392e+09 1.5799e+12 12861
## - total_snow          1 4.0712e+09 1.5804e+12 12862
## <none>                             1.5763e+12 12862
## - snow_grnd           1 1.4497e+10 1.5908e+12 12865
## - mean_temp           1 1.8454e+10 1.5948e+12 12867
## - unemployability     1 3.6625e+10 1.6130e+12 12874
## - cpi                 1 7.3905e+10 1.6502e+12 12887
## 
## Step:  AIC=12861.28
## visits_by_day ~ mean_temp + cpi + unemployability + total_snow + 
##     snow_grnd
## 
##                   Df  Sum of Sq        RSS   AIC
## - total_snow       1 3.3969e+09 1.5833e+12 12861
## <none>                          1.5799e+12 12861
## - snow_grnd        1 1.2474e+10 1.5924e+12 12864
## - mean_temp        1 1.5999e+10 1.5959e+12 12865
## - unemployability  1 4.6938e+10 1.6268e+12 12877
## - cpi              1 7.0801e+10 1.6507e+12 12885
## 
## Step:  AIC=12860.55
## visits_by_day ~ mean_temp + cpi + unemployability + snow_grnd
## 
##                   Df  Sum of Sq        RSS   AIC
## <none>                          1.5833e+12 12861
## - mean_temp        1 1.3260e+10 1.5965e+12 12864
## - snow_grnd        1 2.3778e+10 1.6071e+12 12867
## - unemployability  1 5.0480e+10 1.6338e+12 12877
## - cpi              1 7.0689e+10 1.6540e+12 12884
\end{verbatim}

\begin{Shaded}
\begin{Highlighting}[]
\FunctionTok{summary}\NormalTok{(lm.step)}
\end{Highlighting}
\end{Shaded}

\begin{verbatim}
## 
## Call:
## lm(formula = visits_by_day ~ mean_temp + cpi + unemployability + 
##     snow_grnd, data = my_data[, 4:13])
## 
## Residuals:
##    Min     1Q Median     3Q    Max 
## -85945 -40444 -17109  28471 171283 
## 
## Coefficients:
##                   Estimate Std. Error t value Pr(>|t|)    
## (Intercept)     718249.586 133670.238   5.373 1.12e-07 ***
## mean_temp          639.199    288.292   2.217  0.02699 *  
## cpi              -5353.926   1045.820  -5.119 4.16e-07 ***
## unemployability      6.135      1.418   4.326 1.78e-05 ***
## snow_grnd         1044.792    351.882   2.969  0.00311 ** 
## ---
## Signif. codes:  0 '***' 0.001 '**' 0.01 '*' 0.05 '.' 0.1 ' ' 1
## 
## Residual standard error: 51930 on 587 degrees of freedom
##   (3 observations deleted due to missingness)
## Multiple R-squared:  0.08057,    Adjusted R-squared:  0.0743 
## F-statistic: 12.86 on 4 and 587 DF,  p-value: 4.843e-10
\end{verbatim}

\begin{Shaded}
\begin{Highlighting}[]
\NormalTok{test\_data }\OtherTok{\textless{}{-}} \FunctionTok{read.csv}\NormalTok{(}\StringTok{"Data/testing.csv"}\NormalTok{)}
\NormalTok{lm.full }\OtherTok{=} \FunctionTok{lm}\NormalTok{(value }\SpecialCharTok{\textasciitilde{}}\NormalTok{ ., }\AttributeTok{data =}\NormalTok{ test\_data[,}\DecValTok{3}\SpecialCharTok{:}\DecValTok{10}\NormalTok{])}
\NormalTok{lm.step }\OtherTok{=} \FunctionTok{step}\NormalTok{(lm.full, }\AttributeTok{direction =} \StringTok{"backward"}\NormalTok{)}
\end{Highlighting}
\end{Shaded}

\begin{verbatim}
## Start:  AIC=11013.75
## value ~ max_temp + mean_temp + min_temp + spd_max_gust + total_precip + 
##     total_rain + total_snow
## 
##                Df  Sum of Sq        RSS   AIC
## - spd_max_gust  1 7.9276e+11 4.5679e+14 11012
## <none>                       4.5600e+14 11014
## - mean_temp     1 3.5671e+12 4.5957e+14 11015
## - max_temp      1 6.7372e+12 4.6274e+14 11018
## - min_temp      1 7.9528e+12 4.6395e+14 11019
## - total_snow    1 1.7101e+14 6.2701e+14 11138
## - total_precip  1 1.8573e+14 6.4173e+14 11147
## - total_rain    1 2.3401e+14 6.9001e+14 11176
## 
## Step:  AIC=11012.44
## value ~ max_temp + mean_temp + min_temp + total_precip + total_rain + 
##     total_snow
## 
##                Df  Sum of Sq        RSS   AIC
## <none>                       4.5679e+14 11012
## - mean_temp     1 3.5401e+12 4.6033e+14 11014
## - max_temp      1 6.9012e+12 4.6369e+14 11016
## - min_temp      1 7.8021e+12 4.6460e+14 11017
## - total_snow    1 1.7223e+14 6.2902e+14 11137
## - total_precip  1 1.8495e+14 6.4174e+14 11145
## - total_rain    1 2.3476e+14 6.9155e+14 11175
\end{verbatim}

\begin{Shaded}
\begin{Highlighting}[]
\FunctionTok{summary}\NormalTok{(lm.step)}
\end{Highlighting}
\end{Shaded}

\begin{verbatim}
## 
## Call:
## lm(formula = value ~ max_temp + mean_temp + min_temp + total_precip + 
##     total_rain + total_snow, data = test_data[, 3:10])
## 
## Residuals:
##      Min       1Q   Median       3Q      Max 
## -3365753  -683300  -216486   395390  4807943 
## 
## Coefficients:
##                Estimate Std. Error t value Pr(>|t|)    
## (Intercept)  2732523.26  513290.01   5.324 1.72e-07 ***
## max_temp      -36584.59   15091.05  -2.424   0.0158 *  
## mean_temp      41667.21   23997.75   1.736   0.0833 .  
## min_temp      -29268.23   11354.67  -2.578   0.0103 *  
## total_precip     555.97      44.30  12.550  < 2e-16 ***
## total_rain      -864.41      61.14 -14.139  < 2e-16 ***
## total_snow      -924.27      76.32 -12.111  < 2e-16 ***
## ---
## Signif. codes:  0 '***' 0.001 '**' 0.01 '*' 0.05 '.' 0.1 ' ' 1
## 
## Residual standard error: 1084000 on 389 degrees of freedom
## Multiple R-squared:  0.4403, Adjusted R-squared:  0.4317 
## F-statistic: 51.01 on 6 and 389 DF,  p-value: < 2.2e-16
\end{verbatim}

\begin{Shaded}
\begin{Highlighting}[]
\NormalTok{test\_model\_random }\OtherTok{\textless{}{-}} \FunctionTok{plm}\NormalTok{(value }\SpecialCharTok{\textasciitilde{}}\NormalTok{ mean\_temp }\SpecialCharTok{+}\NormalTok{ total\_rain }\SpecialCharTok{+}\NormalTok{ total\_snow, }\AttributeTok{data =}\NormalTok{ test\_data, }\AttributeTok{index =} \FunctionTok{c}\NormalTok{(}\StringTok{"prov"}\NormalTok{,}\StringTok{"Date"}\NormalTok{), }\AttributeTok{model =} \StringTok{"within"}\NormalTok{)}
\FunctionTok{summary}\NormalTok{(test\_model\_random)}
\end{Highlighting}
\end{Shaded}

\begin{verbatim}
## Oneway (individual) effect Within Model
## 
## Call:
## plm(formula = value ~ mean_temp + total_rain + total_snow, data = test_data, 
##     model = "within", index = c("prov", "Date"))
## 
## Balanced Panel: n = 3, T = 132, N = 396
## 
## Residuals:
##     Min.  1st Qu.   Median  3rd Qu.     Max. 
## -1965617  -470003  -131930   392193  3753106 
## 
## Coefficients:
##              Estimate Std. Error t-value Pr(>|t|)  
## mean_temp  14025.7966  7954.2738  1.7633  0.07863 .
## total_rain    18.0534     9.6786  1.8653  0.06289 .
## total_snow    52.6931    44.2409  1.1910  0.23436  
## ---
## Signif. codes:  0 '***' 0.001 '**' 0.01 '*' 0.05 '.' 0.1 ' ' 1
## 
## Total Sum of Squares:    2.7805e+14
## Residual Sum of Squares: 2.7241e+14
## R-Squared:      0.020289
## Adj. R-Squared: 0.0077281
## F-statistic: 2.69213 on 3 and 390 DF, p-value: 0.045927
\end{verbatim}

\begin{Shaded}
\begin{Highlighting}[]
\NormalTok{test\_model }\OtherTok{\textless{}{-}} \FunctionTok{lm}\NormalTok{(value }\SpecialCharTok{\textasciitilde{}}\NormalTok{ max\_temp }\SpecialCharTok{+}\NormalTok{ total\_rain }\SpecialCharTok{+}\NormalTok{ total\_snow , }\AttributeTok{data=}\NormalTok{test\_data)}
\FunctionTok{summary}\NormalTok{(test\_model)}
\end{Highlighting}
\end{Shaded}

\begin{verbatim}
## 
## Call:
## lm(formula = value ~ max_temp + total_rain + total_snow, data = test_data)
## 
## Residuals:
##      Min       1Q   Median       3Q      Max 
## -2154453  -925505  -223686   631443  4809900 
## 
## Coefficients:
##               Estimate Std. Error t value Pr(>|t|)    
## (Intercept) 4918431.76  378097.41  13.008  < 2e-16 ***
## max_temp     -37008.35   11528.71  -3.210  0.00144 ** 
## total_rain     -104.70      11.56  -9.057  < 2e-16 ***
## total_snow     -292.59      57.90  -5.054 6.66e-07 ***
## ---
## Signif. codes:  0 '***' 0.001 '**' 0.01 '*' 0.05 '.' 0.1 ' ' 1
## 
## Residual standard error: 1281000 on 392 degrees of freedom
## Multiple R-squared:  0.2115, Adjusted R-squared:  0.2054 
## F-statistic: 35.04 on 3 and 392 DF,  p-value: < 2.2e-16
\end{verbatim}

\hypertarget{results}{%
\subsubsection{Results}\label{results}}

\hypertarget{conclusion}{%
\subsubsection{Conclusion}\label{conclusion}}

\end{document}
